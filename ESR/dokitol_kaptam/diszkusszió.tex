\section{Diszkusszió}

A gyakorlat során megmértük Cr$^{3+}$ illetve Mn$^{2+}$ ionok ESR-spektrumát. A Cr$^{3+}$-minta $90.50\%$ $I=0$ spinű magot ($^{50}$Cr, $^{52}$Cr, $^{54}$Cr), míg $9.50\%$-ban $I=3/2$ spinű magot ($^{53}$Cr) tartalmaz. Ennek megfelelően a spektrumban egy jól látható csúcs jelent meg. Az $I=3/2$ magspin hatására a hiperfinom-felhasadás következtében megjelenő 4 kisebb csúcs épphogy láthatóan jelenik csak meg (a magok jóval kisebb száma és a négyfelé hasadás következtében), amely alapján jól becsülhető volt a Cr$^{3+}$-ra jellemző hiperfinom kölcsönhatási állandó $A=$. 
A Mn$^{2+}$-minta ESR spektrumám 6, egymástól azonos távolságú csűcsot figyeltünk meg, amely megfelel annak, hogy a minta csupán $^{55}$Mn magot tartalmaz, amelyre $I= 5/2$. A Cr$^{3+}$-minta méréséből meghatározott frekvencia birtokában a Mn$^{2+}$-ra a giromágneses faktor $g=$-nak adódott. A hat csúcs helyére egyenest illsztve pedig a finom kölcsönhatási állandót kaptuk meg, melyre $A=$ adódott.